\section{Number Theory}
    \subsection{Divisors}
       \subsubsection{formulas}
           \includeCode{c++} {number of divisors}{lhstyle}{divisors/num_div.cpp}
           \includeCode{c++} {sum of divisors}{lhstyle}{divisors/sum_div.cpp}
    \subsection{Primes}
        \includeCode{c++} {prime factorization}{lhstyle}{factorization/pf.cpp}
        \includeCode{c++} {number of co-primes with n}{lhstyle}{euler/eulerTotient.cpp}
        \includeCode{c++} {Prime Check}{lhstyle}{primes.cpp}


    \subsection{Math}
        \subsubsection{Vieta's Formulas for a Polynomial of Degree \( n \)}

        \textbf{Problem}: Given a polynomial of degree \( n \):

        \[
        P(x) = a_n x^n + a_{n-1} x^{n-1} + \cdots + a_1 x + a_0
        \]

        with roots \( r_1, r_2, \ldots, r_n \), express the sums and products of its roots using Vieta's formulas.

        \textbf{Solution}:
        Using Vieta's formulas, we can relate the coefficients of the polynomial to sums and products of its roots:

        \begin{itemize}
            \item Sum of the roots taken one at a time:
            \[
            r_1 + r_2 + \cdots + r_n = -\frac{a_{n-1}}{a_n}
            \]
            \item Sum of the products of the roots taken two at a time:
            \[
            r_1r_2 + r_1r_3 + \cdots + r_{n-1}r_n = \frac{a_{n-2}}{a_n}
            \]
            \item Sum of the products of the roots taken three at a time:
            \[
            r_1r_2r_3 + r_1r_2r_4 + \cdots + r_{n-2}r_{n-1}r_n = -\frac{a_{n-3}}{a_n}
            \]
            \item Continue this pattern until:
            \item Product of the roots (for even \( n \)):
            \[
            r_1 r_2 \cdots r_n = (-1)^n \frac{a_0}{a_n}
            \]
        \end{itemize}

        \subsubsection*{Example Problem Using Vieta's Formulas}

        \textbf{Problem}: Given a quadratic equation \(x^2 + bx + c = 0\) with roots \(r_1\) and \(r_2\), find \(r_1 + r_2\) and \(r_1r_2\).

        \textbf{Solution}:
        Using Vieta's formulas for a quadratic equation \(ax^2 + bx + c = 0\):

        \begin{itemize}
            \item Sum of the roots:
            \[
            r_1 + r_2 = -\frac{b}{a}
            \]
            \item Product of the roots:
            \[
            r_1r_2 = \frac{c}{a}
            \]
        \end{itemize}

        For the given quadratic \(x^2 + bx + c = 0\) (where \(a = 1\)):

        \textbf{Also:}
        To find the roots of the quadratic equation, we can use the discriminant formula, \( D = b^2 - 4ac \). The roots will then be \( x_1 = \frac{b - \sqrt{D}}{2} \) and \( x_2 = \frac{b + \sqrt{D}}{2} \).


        \begin{itemize}
            \item \(r_1 + r_2 = -b\)
            \item \(r_1r_2 = c\)
        \end{itemize}

        \subsubsection*{Example Vieta's Formula for Cubic Equation}

        When considering a cubic equation in the form of \( f(x) = ax^3 + bx^2 + cx + d \), Vieta's formula states that if the equation \( f(x) = 0 \) has roots \( r_1, r_2, \) and \( r_3 \), then:

        \begin{itemize}
            \item \( r_1 + r_2 + r_3 = -\frac{b}{a} \)
            \item \( r_1r_2 + r_2r_3 + r_3r_1 = \frac{c}{a} \)
            \item \( r_1r_2r_3 = -\frac{d}{a} \)
        \end{itemize}